\chapter{Note e conclusioni}

% **************************** Define Graphics Path **************************
\ifpdf
    \graphicspath{{Chapter4/Figs/Raster/}{Chapter4/Figs/PDF/}{Chapter3/Figs/}}
\else
    \graphicspath{{Chapter4/Figs/Vector/}{Chapter4/Figs/}}
\fi

\section{Note per l'esecuzione}
\label{sec:note}
Il documento è corredato da 5 files:

\begin{itemize}
  \item \texttt{bus\_model.mod}, che contiene il modello del problema codificato in AMPL, opportunamente commentato;
  \item \texttt{bus\_dat.dat}, che contiene i dati del problema come descritti in in §\ref{cha:descrizione};
  \item \texttt{decide\_time.run}, eseguibile per verificare il costo minimo al variare del parametro \textit{t} nell'intervallo [6,21] (vale a dire che il suono della campanella varia tra le 7:30 e le 8:45, a scatti di 5 minuti);
  \item \texttt{decide\_time.sa1}, script utile per implementare il ciclo necessario per lo script \texttt{decide\_time.run} al punto sopra; 
  \item \texttt{display\_solution.run}, eseguibile per mostrare la soluzione ottima dato \textit{t =10}.
\end{itemize}

Per eseguire \texttt{display\_solution.run} o \texttt{decide\_time.run} è sufficiente aprire AMPL da riga di comando e digitare il comando \texttt{include display\_solution.run} o \texttt{include decide\_time.run}. L'ordine con cui vengono eseguiti questi due comandi è ininfluente.

\section{Conclusioni}

Il problema viene risolto con il metodo del branch and bound. Nella soluzione ottima vengono impiegati i bus $1$ (omologato per 18 posti) e $2$ (26 posti). Il costo minimo individuato ammonta a €$21680$ a semestre.
Il percorso individuato per il bus n° 1 è $6-3-8$, quello individuato per il bus 2 è $7-5-2-8$, mentre il bus $3$ rimane fermo al deposito $6$.
In particolare questa è l'assegnazione di studenti per fermata:

\begin{itemize}
\item fermata 2: studenti $10$, $11$, $13$, $14$, $15$ e $22$;
\item fermata 3: studenti $1$, $4$, $5$, $6$, $7$, $8$, $9$, $18$ e $19$;
\item fermata 5: studenti $2$, $3$, $12$, $16$, $17$, $20$ e $21$.
\end{itemize}

Anticipando l'orario di inizio lezioni alle 7:30 o 7:35 ($t=6$ o $t=7$) il costo minimo semestrale per il servizio trasporti corrisponde a €$25040$. Fissando l'orario di inizio in un orario compreso tra le 7:40 e le 8:45 è possibile soddisfare i vincoli spendendo €$21680$ a semestre.


