%!TEX root = ../thesis.tex
%*******************************************************************************
%****************************** Third Chapter *********************************
%*******************************************************************************

\chapter{Modellazione}

\ifpdf
    \graphicspath{{Chapter3/Figs/Raster/}{Chapter3/Figs/PDF/}{Chapter3/Figs/}}
\else
    \graphicspath{{Chapter3/Figs/Vector/}{Chapter3/Figs/}}
\fi
\section{Elementi del modello}
Di seguito vengono presentati gli insiemi, i parametri e le variabili individuate per la successiva formulazione della funzione obiettivo e dei vincoli.
\subsection{Insiemi}
Nel testo sono stati individuati i seguenti insiemi, nei quali spazia il dominio delle variabili definite in §\ref{subsec:var}:
\begin{itemize}
  \item $S = \{1,\, 2,\, \ldots,\, 22\}$ : studenti afferenti al servizio scuolabus;%TODO modify
  \item $F = \{1,\, 2,\, 3,\, 4,\, 5\}$ : insieme dei 4 punti di raccolta in cui poter prelevare studenti;
  \item $D = \{6,\, 7\}$ : insieme dei 2 depositi per scuolabus;
  \item $I = \{8\}$ : insieme delle scuole presenti (nel problema ve ne è una sola);
  \item $B = \{1,\,2,\,3\}$ : flotta degli scuolabus.
\end{itemize}
Per convenienza definiamo anche $L = F \cup D \cup I$.

\subsection{Parametri}

I costi variabili di una soluzione vengono calcolati sulla base delle distanze e dei tempi di percorrenza tra archi di un grafo completo. I seguenti parametri modellano la matrici delle distanze e dei tempi:
\begin{itemize}
  \item $DistMatr_{ij}$ con $i,j \in L$: matrice delle distanze in chilometri tra i nodi del grafo delle fermate in $L$;
  \item $TimeMatr_{ij}$ con $i,j \in L$: matrice dei tempi di percorrenza medi in minuti tra i nodi del grafo delle fermate in $L$;
\end{itemize}
L'assegnazione di studenti alle fermate e ai bus deve considerare anche le disponibilità degli studenti riguardo alle fermate e le capacità dei bus. Inoltre un bus non visita una fermata se in questa non preleva un numero minimo di studenti:
\begin{itemize}
\item $FeStMatr_{is}$ con $s \in S$ e $i \in F$: matrice che indica per ogni studente le fermate nelle quali è disponibile a prendere il bus;
\item $C_k$ con $k \in B$: capacità effettiva di ogni bus, in posti a sedere;
\item $minS$ : numero minimo di studenti che un bus preleva in una fermata.
\end{itemize}
Ogni bus deve essere assegnato al proprio deposito:
\begin{itemize}
  \item $b_{ik}$ con $i \in D$ e $k \in B$: il bus $k$ parte dal deposito $i$;
\end{itemize}
L'orario in cui suona la campanella determina la durata massima di una corsa del bus:
\begin{itemize}
  \item $t$ : minuti compresi tra le ore 7:00 e l'orario di inizio delle lezioni, diviso 5 (es. se le lezioni iniziano alle 7:50 si ha $t = 10$).
\end{itemize}


\subsection{Variabili}
\label{subsec:var}
Un'assegnazione alle seguenti variabili che rispetta i vincoli imposti in §\ref{sec:constraints} rappresenta una candidata soluzione al problema:
\begin{itemize}
\item $x_{kij}$ con $k \in B$, $i,j \in L$: indica se il bus $k$ deve percorrere il tragitto che collega $i$ e $j$;
\item $y_{ki}$ con $i \in L$, $k \in B$: indica se il bus $k$ si deve fermare in $i$;
\item $z_{ski}$ con $i \in L$, $k \in B$: indica se il bus $k$ deve prelevare lo studente $s$ nella fermata $i$.
\end{itemize}


\section{Funzione obiettivo}
Il problema richiede di individuare una soluzione di costo minimo su base semestrale. Si suppone per semplicità che un semestre abbia 120 giorni lavorativi, e non si considerano nel calcolo i costi del deposito diurno e notturno relativi ai giorni non lavorativi (in quanto inevitabili e aggiungono una quota fissa alla funzione obiettivo indipendentemente da quanti bus vengono utilizzati). 

I costi presenti nel testo del problema si suddividono in:
\begin{itemize}
  \item \textbf{costi variabili \textit{dipendenti dalla distanza}}, che consistono nella somma tra quota manutenzione (€0,10 al km) e quota carburante (€0,20 al km) moltiplicati per il numero di km percorsi al semestre (considerando 2 tratte al giorno per 120 giorni, per ogni bus):
\begin{equation}
  \label{eq:vardist}
  (0.1 + 0.2) * 2 * 120\sum_{k \in B}\sum_{i \in L}\sum_{j \in L}\Big(DistMatr_{ij} * x_{kij}\Big)
\end{equation}
  \item \textbf{costi variabili \textit{dipendenti dal tempo impiegato}}, che sono pari alla paga al minuto per autista moltiplicata per il numero di minuti maturate per semestre (considerando 2 tratte al giorno per 120 giorni, per ogni bus):
\begin{equation}
  \label{eq:vartime}
  \frac{30}{60} * 2 * 120\Bigg(\sum_{k \in B}\sum_{i \in L}\sum_{j \in L}\Big(TimeMatr_{ij} * x_{kij}\Big) + 2\sum_{k \in B}\sum_{i \in F} y_{ki} \Bigg)
\end{equation}
\item \textbf{costi fissi per autobus \textit{impiegato}}, che ammontano alla somma tra i costi semestrali di assicurazione e bollo (€3200), la somma dei costi giornalieri per l'utilizzo esclusivamente notturno del deposito (€15, per ognuno dei 120 giorni lavorativi):
\begin{equation}
  \label{eq:fixedused}
  (15 * 120 + 3200) \sum_{k \in B}\sum_{i \in D}\sum_{j \in F} x_{kij}
\end{equation}
\item \textbf{costi fissi per autobus \textit{non impiegato}}, che sono pari al numero di giorni lavorativi (120) moltiplicato per il costo dell'utilizzo diurno del deposito (€25) moltiplicato per il numero di bus inutilizzati:
\begin{equation}
  \label{eq:fixedunused}
    25 * 120 \Big(|B| - \sum_{k \in B}\sum_{i \in D}\sum_{j \in F} x_{kij}\Big)
  \end{equation}
\end{itemize}
La funzione obiettivo è data da (\ref{eq:vardist}) + (\ref{eq:vartime}) + (\ref{eq:fixedused}) + (\ref{eq:fixedunused}).



\section{Vincoli}
\label{sec:constraints}
%Ogni fermata viene visitata al più una volta.
%\begin{equation}
%    \label{eq:maxVisitStop}
%    \sum_{k \in B}y_{ik} \leq 1, \quad \forall i \in L
%\end{equation}
Un nodo $i$ corrispondente a un punto di raccolta studenti (ossia appartenente a $F$) è un nodo \textit{di transito}, nel quale non può iniziare o terminare una corsa di bus: il numero di bus che arriva in $i$ dopo aver visitato un nodo adiacente deve essere uguale al numero di bus che visitano un arco adiacente subito dopo aver visitato $i$:
\begin{equation}
    \sum_{j \in F \cup I} x_{kij} = \sum_{j \in F \cup D} x_{kji},\quad \forall k \in B,\;\forall i \in F
\end{equation}
%Un nodo corrispondente a una scuola può essere l'ultimo nodo visitato da un bus, dunque il numero di partenze da un nodo scuola è sicuramente inferiore a quello degli arrivi in esso:
%\begin{equation}
%  \label{EnterExitInequalitySchool}
%  \sum_{j \in I} x_{kij} \leq \sum_{j \in I} x_{kji},\quad \forall k \in B,\;\forall i \in I
%\end{equation}
Nessun bus arriva in un deposito, né rimane fermo nella stessa locazione, né visita altre fermate per prelevare studenti dopo essere giunto alla scuola, né collega direttamente un deposito a una scuola.
\begin{equation}
  \label{eq:forbiddenMovements}
    \sum_{k \in B}\sum_{i \in L}(x_{kii} + \sum_{j \in D}x_{kij}) + \sum_{k \in B}\sum_{i \in I}\sum_{j \in L}x_{kij} + \sum_{k \in B}\sum_{i \in D}\sum_{j \in I}x_{kij} = 0
\end{equation}
In ogni nodo che non può costituire punto di partenza della corsa di un bus (quindi in ogni punto di raccolta e in ogni scuola), il numero di bus che arriva in $i$ è pari a 1 se e solo se $i$ viene visitato, altrimenti è pari a 0.
\begin{equation}
  \sum_{i \in F \cup D} x_{kij} = y_{kj},\quad \forall i \in F \cup I,\; \forall k \in B
\end{equation}
In ogni nodo che non può costituire punto di termine della corsa di un bus (quindi in ogni punto di raccolta e in ogni deposito), il numero di bus che parte da $i$ è pari a 1 se e solo se $i$ viene visitato, altrimenti è pari a 0.
\begin{equation}
  \sum_{j \in F \cup I} x_{kij} = y_{ki},\quad \forall i \in F \cup D,\; \forall k \in B
\end{equation}
Ogni studente deve essere prelevato una sola volta, in una sola fermata.
\begin{equation}
    \sum_{k \in B}\sum_{i \in F} z_{ski} = 1,\quad \forall s \in S
\end{equation}
Uno studente può prendere un bus $k$ nella fermata $i$ solo se il bus $k$ si ferma in $i$.
\begin{equation}
    z_{ski} \leq y_{ki},\quad \forall s \in S,\;\forall k \in B,\;\forall i \in F
\end{equation}
Se uno studente non è disposto a prendere il bus nella fermata $i$ allora non deve essere prelevato in $i$.
\begin{equation}
    \sum_{k \in B} z_{ski} \leq FeStMatr_{is},\quad \forall i \in F,\;\forall s \in S
\end{equation}
Per ogni bus l'occupazione massima non può essere superiore al numero di posti disponibili. Poiché una volta raggiunta la scuola un bus non può più caricare studenti, la sua \textit{occupazione massima} è pari al numero di studenti presenti nel bus al momento dell'arrivo a scuola, ovvero alla somma di tutti gli studenti caricati nel bus in tutte le fermate visitate. 
\begin{equation}
    \sum_{s \in S}\sum_{i \in F}z_{ski} \leq C_{k},\quad \forall k \in B
\end{equation}
Se una fermata viene visitata dal bus $k$ allora il bus non è rimasto fermo al deposito.
\begin{equation}
    x_{kij} \leq \sum_{l \in D}\sum_{m \in F}x_{klm}, \quad \forall b \in B,\;i \in F \cup D,\;\forall j \in F
\end{equation}
Un bus parte dal proprio deposito, dunque non percorre archi adiacenti a un deposito non suo.
\begin{equation}
  \sum_{j \in L}x_{kij} \leq b_{ik}, \quad \forall k \in B,\;\forall i \in D
\end{equation}
Ogni bus partito da un deposito termina la propria corsa in una scuola.
\begin{equation}
  \sum_{k \in B}\sum_{i \in D}\sum_{j \in F} x_{kij} = \sum_{k \in B}\sum_{i \in F}\sum_{j \in I} x_{kij}
\end{equation}
Ogni bus deve completare la propria corsa entro il suono della campanella (considerando 2 minuti di stazionamento per ogni fermata, escluse la partenza e l'arrivo della corsa).
\begin{equation}
  \sum_{i \in L}\sum_{j \in L}\Big(TimeMatr_{ij} * x_{kij}\Big) + 2\sum_{i \in F} y_{ik} \leq 5t, \quad \forall k \in B
\end{equation}
Un bus visita una fermata solo se può prelevare un numero minimo di studenti.
\begin{equation}
  y_{ki} \leq \frac{1}{minS}\sum_{s \in S}z_{ski}, \quad \forall i \in F,\;\forall k \in B
\end{equation}
Le variabili $x_{kij}$, $y_{ik}$, $z_{ski}$ sono binarie per ogni valore che possono assumere $i$, $j$, $k$, $s$.
\begin{equation}
    x_{kij} \in \{0,1\},\quad \forall i \in L,\;\forall j \in L,\;\forall k \in B
\end{equation}
\begin{equation}
    y_{ki} \in \{0,1\},\quad \forall k \in B,\;\forall i \in L
\end{equation}
\begin{equation}
    z_{ski} \in \{0,1\},\quad \forall s \in S,\;\forall k \in B,\;\forall i \in F
\end{equation}

\section{Studio della presenza di cicli}
Il vincolo (\ref{eq:forbiddenMovements}) impedisce a un bus di arrivare in un deposito. Ciò implica che è impossibile che un percorso assegnato a un bus contenga un ciclo che interessi un deposito.

La binarietà di $y$ implica che un punto di raccolta studenti possa essere attraversato al più una volta. In un ciclo almeno un nodo deve essere raggiunto due volte da un bus: dunque nel modello sopra formulato qualsiasi ciclo deve coinvolgere una scuola. Inoltre il vincolo (\ref{eq:forbiddenMovements}) impedisce che un bus prelevi altri studenti dopo avere raggiunto una scuola: quindi un ciclo può coinvolgere solo scuole. Nel testo del problema è prevista una sola scuola: è dunque impossibile che si verifichino cicli.

