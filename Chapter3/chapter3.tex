%!TEX root = ../thesis.tex
%*******************************************************************************
%****************************** Third Chapter *********************************
%*******************************************************************************

\chapter{Modellazione}

\ifpdf
    \graphicspath{{Chapter3/Figs/Raster/}{Chapter3/Figs/PDF/}{Chapter3/Figs/}}
\else
    \graphicspath{{Chapter3/Figs/Vector/}{Chapter3/Figs/}}
\fi
Quando si incarica un bus di percorrere l'arco $ij$ il bus si ferma a prelevare studenti in $i$ e $j$. Se le strade si $i,j$ fossero un sottoinsieme delle strade di un altro arco $k,l$ e si volesse far fermare il bus in $k,l$ e non in $i,j$ allora l'arco $i,j$ non risulta assegnato al bus.
\section{Analisi del problema}
Nessun nodo $i$ corrispondente a una fermata di un bus (ossia appartenente a $A$) è un nodo di inizio o di termine di una corsa del bus: il numero di bus che arriva in $i$ dopo aver visitato un nodo adiacente è pari al numero di bus che visitano un arco adiacente subito dopo aver visitato $i$:

Il suddetto numero di bus che entra in (o esce da) $i$ è pari a 1 se se solo se il bus si ferma in $i,j$, altrimenti è pari a 0.


Ogni studente deve essere prelevato una e una sola volta volta, in una e una sola fermata.

Nessuno studente viene prelevato solo

\section{Vincoli}

