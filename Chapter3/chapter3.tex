%!TEX root = ../thesis.tex
%*******************************************************************************
%****************************** Third Chapter *********************************
%*******************************************************************************

\chapter{Modellazione}

\ifpdf
    \graphicspath{{Chapter3/Figs/Raster/}{Chapter3/Figs/PDF/}{Chapter3/Figs/}}
\else
    \graphicspath{{Chapter3/Figs/Vector/}{Chapter3/Figs/}}
\fi
Quando si incarica un bus di percorrere l'arco $ij$ il bus si ferma a prelevare studenti in $i$ e $j$. Se le strade si $i,j$ fossero un sottoinsieme delle strade di un altro arco $k,l$ e si volesse far fermare il bus in $k,l$ e non in $i,j$ allora l'arco $i,j$ non risulta assegnato al bus.
\section{Analisi del problema}


\section{Vincoli}
Nessun nodo $i$ corrispondente a una fermata di un bus (ossia appartenente a $A$) è un nodo di inizio o di termine di una corsa del bus: il numero di bus che arriva in $i$ dopo aver visitato un nodo adiacente è pari al numero di bus che visitano un arco adiacente subito dopo aver visitato $i$:
\begin{equation}
    \sum_{j \in L} x_{kij} = \sum_{j \in L} x_{kji},\quad \forall k \in B,\;\forall i \in L
\end{equation}
Il suddetto numero di bus che arriva in (o parte da) $i$ è pari a 1 se e solo se il bus si ferma in $i,j$, altrimenti è pari a 0.
\begin{equation}
    \sum_{j \in L} x_{kij} = y_{ik},\quad \forall i \in L,\; \forall k \in B 
\end{equation}
Ogni studente deve essere prelevato una e una sola volta volta, in una e una sola fermata.
\begin{equation}
    \sum_{k \in B}\sum_{i \in L} z_{ski} = 1,\quad \forall s \in S
\end{equation}
Uno studente può prendere un bus $k$ nella fermata $i$ solo se il bus $k$ si ferma in $i$.
\begin{equation}
    z_{ski} \leq y_{ik},\quad \forall s \in S,\;\forall k \in B,\;\forall i \in F
\end{equation}
Se uno studente non è disposto a prendere il bus nella fermata $i$ allora non deve essere prelevato in $i$.
\begin{equation}
    \sum_{k \in B} z_{ski} \leq FSMatrix_{is},\quad \forall s \in S,\;\forall i \in F
\end{equation}
Per ogni bus l'occupazione massima non può essere superiore al numero di posti disponibili. Poiché una volta raggiunta la scuola un bus non può più caricare studenti, la sua \textit{occupazione massima} è pari al numero di studenti presenti nel bus al momento dell'arrivo a scuola, ovvero alla somma di tutti gli studenti caricati nel bus in tutte le fermate visitate. 
\begin{equation}
    \sum_{s \in S}\sum_{i \in F}z_{ski} \leq C_{k},\quad \forall k \in B
\end{equation}
Nessun bus arriva in un deposito, né rimane fermo nella stessa locazione, né visita altre fermate per prelevare studenti \footnote{Ma potrebbe visitare altre scuole, qualora ve ne fossero}.
\begin{equation}
    \sum_{k \in B}\sum_{i \in L}(x_{kii} + \sum_{i \in D}x_{kij}) + \sum_{k \in B}\sum_{j \in I}\;\sum_{i \in F \cup D}x_{kij} = 0
\end{equation}
Se una fermata viene visitata dal bus $k$ allora il bus non è rimasto fermo al deposito.
\begin{equation}
    \sum_{i \in F \cup D}x_{kij} \leq \sum_{i \in D}\sum_{l \in A}x_{kil}, \quad \forall b \in B,\;\forall j \in F
\end{equation}
Ogni fermata viene visitata al più una volta.
\begin{equation}
    \sum_{b \in B}y_{ki} \leq 1, \quad \forall i \in L
\end{equation}
Un bus parte dal proprio deposito, dunque non percorre archi adiacenti a un deposito non suo.
\begin{equation}
    \sum_{j \in L}x_{kij} \leq b_{ik}, \quad \forall k \in B,\;\forall i \in L
\end{equation}
Le variabili $x_{kij}$, $y_{ik}$, $z_{ski}$ sono binarie per ogni valore che possono assumere $i$, $j$, $k$, $s$.
\begin{equation}
    x_{kij} \in \{0,1\},\quad \forall i \in L,\;\forall j \in L,\;\forall k \in B
\end{equation}
\begin{equation}
    y_{ik} \in \{0,1\},\quad \forall i \in L,\;\forall k \in B
\end{equation}
\begin{equation}
    z_{ski} \in \{0,1\},\quad \forall s \in S,\;\forall k \in B,\;\forall i \in F
\end{equation}

