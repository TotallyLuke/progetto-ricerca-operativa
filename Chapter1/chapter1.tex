%!TEX root = ../thesis.tex
%*******************************************************************************
%*********************************** First Chapter *****************************
%*******************************************************************************

\chapter{Prefazione}  %Title of the First Chapter

\ifpdf
    \graphicspath{{Chapter1/Figs/Raster/}{Chapter1/Figs/PDF/}{Chapter1/Figs/}}
\else
    \graphicspath{{Chapter1/Figs/Vector/}{Chapter1/Figs/}}
\fi

%********************************** First Section ******************************

Il seguente progetto mira a risolvere un problema di ricerca operativa attraverso la programmazione lineare. Il problema affrontato non è un caso reale, ma è stato ideato per il progetto cercando di simulare al meglio la realtà.

Il caso trattato riguarda la pianificazione comunale del servizio di trasporto studenti, in un Comune nel quale sono presenti più depositi di scuolabus e un solo istituto. I dettagli del problema sono descritti nella sezione §\ref{cha:descrizione} - Descrizione. I parametri numerici sono volutamente bassi per rispettare i limiti imposti dalla versione "demo" di AMPL.

Il presente elaborato è accompagnato da un file \textit{.dat}, un file \textit{.mod} e un file \textit{.run}.